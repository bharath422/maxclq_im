\section{Introduction}
\label{sec:intro}
%\vspace{-6pt}

A clique in an undirected graph is a subset of vertices in which every two vertices
are adjacent to each other. The {\em maximum} clique problem seeks to find 
a clique of the largest possible size in a given graph.

The maximum clique problem, and the related {\em maximal} clique and 
clique {\em enumeration} problems, find applications in a wide variety of domains,
many intimately related to the World Wide Web. 
A few examples include:  
information retrieval \cite{Augustson:1970:AGT:321607.321608}, 
community detection in networks \cite{Fortunato_2010,cite-key,5586496}, 
spatial data mining \cite{wang2009order},
data mining in bioinformatics \cite{19566964},
disease classification based on symptom correlation \cite{Bonner:1964:CT:1662386.1662389}, 
pattern recognition \cite{1211348},
analysis of financial networks \cite{RePEc:eee:csdana:v:48:y:2005:i:2:p:431-443},
computer vision \cite{Horaud:1989:SCT:68871.68875}, and
coding theory \cite{brouwer}.
More examples of application areas can be found in 
\cite{Gutin2004,citeulike:4058448}.

To get a sense for how clique computation arises in the aforementioned contexts, 
consider a generic data mining or information retrieval problem. A typical objective
here is to retrieve data that are considered similar based on some metric. 
Constructing a graph in which vertices correspond to data items and 
edges connect similar items, a clique in the graph would then give a cluster of similar data. 

The maximum clique problem is NP-Hard \cite{Garey:1979:CIG:578533}.
Most exact algorithms for solving it employ some form of {\it branch-and-bound} approach. 
While branching systematically searches for all candidate solutions, bounding (also known as {\em pruning}) discards fruitless candidates based on a previously computed bound. 
The algorithm of Carraghan and Pardalos \cite{pardalos} is an early example of 
a simple and  effective branch-and-bound algorithm for the maximum clique problem.
%are \cite{Bomze99themaximum,pardalos}. 
More recently, \"{O}sterg{\aa}rd \cite{ostergard} introduced 
an improved algorithm and demonstrated its relative advantages via computational experiments. 
Tomita and Seki \cite{citeulike:7905505}, and later, Konc and Jane\v{z}i\v{c} \cite{konc2007improved}
use upper bounds computed using vertex coloring to enhance the branch-and-bound approach. 
%Sewell (1998) presented a maximum clique algorithm designed for dense graphs.
Other examples of branch-and-bound algorithms for the clique problem include 
the works of Bomze et al~\cite{Bomze99themaximum}, Segundo et al.~\cite{SanSegundo}
and Babel and Tinhofer~\cite{babel1990branch}.
Prosser~\cite{prosser2012} in a recent work compares various exact algorithms 
for the maximum clique problem. %An attractive feature of the algorithms of \cite{pardalos} and \cite{ostergard} is their simplicity in terms of ease of implementation. However, their runtimes could  be infeasible for very large graphs.

%only for graphs with up to a few thousand vertices and a corresponding number of edges. 
%Furthermore, both algorithms as well as 
%the algorithms from  \cite{citeulike:7905505} and \cite{konc2007improved}
%are inherently sequential and not naturally parallelizable. 
%The ease with which an algorithm can be parallelized is important 
%for handling large-scale graphs in emerging applications, where graphs with 
%millions (or more) vertices are quite common \cite{kumar:extracting}.

In this paper, we present a new exact branch-and-bound
algorithm for the maximum clique problem that employs 
several new pruning strategies in addition to those used in \cite{pardalos},
\cite{ostergard}, \cite{citeulike:7905505} and \cite{konc2007improved},
making it suitable for massive graphs.
%We also present a heuristic that is based on similar pruning techniques as the exact algorithm but runs much faster---the heuristic follows just one of the ``paths" in the search space, and as a result its complexity is nearly linear-time in the size of the graph, in contrast to the exact algorithm whose worst-case complexity is exponential. 
We run our algorithms on a large variety of test graphs and compare its performance with the
algorithms by Carraghan and Pardalos \cite{pardalos}, \"{O}sterg{\aa}rd \cite{ostergard}, {\bf Tomita et al. \cite{citeulike:7905505}\cite{walcom}, Konc and Jane\v{z}i\v{c} \cite{konc2007improved}, and Segundo et al.~\cite{SanSegundo}}. 
We find our new exact algorithm to be up to orders of magnitude faster on large,
sparse graphs and of comparable runtime on denser graphs.
We also present a hew heuristic, which runs several orders of magnitude faster than the exact algorithm while providing solutions that are optimal or near-optimal for most cases. The algorithms are presented in detail in Section~\ref{sec:algorithms} and the experimental evaluations and comparisons are presented in Section~\ref{sec:experiments}. 

Both the exact algorithm and the heuristic are well-suited for parallelization.
We discuss a simple shared-memory parallelization and present performance 
results showing its promise in Section~\ref{sec:parallelization}. 
We also include (in Section~\ref{sec:applications})
an illustration on how the algorithms can be used as parts of a method
for detecting overlapping communities in networks.
We have made our implementations publicly available\footnote{\url{http://cucis.ece.northwestern.edu/projects/MAXCLIQUE/}}.
%The algorithms are discussed in detail in Section~\ref{sec:algorithms}.

%In Section~\ref{sec:experiments} we present an extensive experimental analysis comparing the performance of our algorithms with the algorithm of Carraghan and Pardalos \cite{pardalos}, the algorithm of \"{O}sterg\.{a}rd \cite{ostergard} and the algorithm of Konc and Jane\v{z}i\v{c} \cite{konc2007improved}.
%The workings of the latter three algorithms is reviewed in Section~\ref{sec:relatedwork}.

%Our testbed includes large-scale real-world graphs drawn from various application domains, large-scale synthetic graphs representing various structures, and DIMACS benchmark graphs.

% \cite{dimacs}, 
%the University of Florida collection, 
%\cite{Davis97theuniversity}, 
% \cite{Chakrabarti:2006:GML:1132952.1132954},
%\vspace{-10pt}

