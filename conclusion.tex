\section{Conclusion}
\label{sec:conclusion}
%\vspace{-10pt}

We presented a new exact and a new heuristic algorithm for the maximum clique problem.
We performed extensive experiments on three broad categories of graphs comparing the 
performance of our algorithms to the algorithms due to
Carraghan and Pardalos (CP) \cite{pardalos},
\"{O}sterg\.{a}rd ({\it cliquer}) \cite{ostergard} and
Konc and Jane\v{z}i\v{c} ({\it MCQD+CS}) \cite{konc2007improved}.
For DIMACS benchmark graphs and certain dense synthetic graphs ({\it rmat\_sd2}), our new exact algorithm performs comparably with the CP algorithm, but slower than {\it cliquer}
and {\it MCQD+CS}. 
For large sparse graphs, both synthetic and real-world, our new algorithm runs
several orders of magnitude faster than the other three. 
The heuristic, which runs many orders of magnitude faster than our exact algorithm and the others, gave optimal solution for 83\% of the test cases, and when it is sub-optimal, its accuracy ranged between 0.83 and 0.99.

In this work, we did not compare the performance of our algorithm against those for which an implementation is not publicly available such as \cite{walcom,AAAI101611}. It would be interesting to implement these and compare in future work. Further, the MCQD implementation uses an adjacency matrix, whereas our algorithm uses an adjacency list to represent the graph. Although it is unlikely for the overall results to be drastically different with a change in the graph representation, it will be interesting to study to what degree the performance will change with
the change in graph representation.
%The exact algorithm was in general  found to be less successful on relatively dense graphs.
%An interesting line of investigation would be to study ways to overcome this.
The heuristic's performance is impressive as presented; still it is worthwhile
to compare with other existing heuristics approaches such as \cite{heu1,heu2}. 
%Another line for future work would be to characterize the class(es) of graphs for which the heuristic is expected to return near-optimal solution.