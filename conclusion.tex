\section{Conclusion}
\label{sec:conclusion}
%\vspace{-10pt}

We presented a new exact and a new heuristic algorithm for the maximum clique problem.
We performed extensive experiments on three broad categories of graphs comparing the 
performance of our algorithms to the algorithms due to
Carraghan and Pardalos (CP) \cite{pardalos},
\"{O}sterg\.{a}rd ({\it cliquer}) \cite{ostergard} and
Konc and Jane\v{z}i\v{c} ({\it MCQD+CS}) \cite{konc2007improved}.
For DIMACS benchmark graphs and certain dense synthetic graphs ({\it rmat\_sd2}), our new exact algorithm performs comparably with the CP algorithm, but slower than {\it cliquer}
and {\it MCQD+CS}. 
For large sparse graphs, both synthetic and real-world, our new algorithm runs
several orders of magnitude faster than the other three. 
And its general runtime is observed to grow nearly linearly with the size of the graphs. 
The heuristic, which runs orders of magnitude faster than our exact algorithm and the others, gave optimal solution for 83\% of the test cases, and when it is sub-optimal, its accuracy ranged between 0.83 and 0.99.
We also showed how the algorithms can be parallelized. Finally, we illustrated a simple application of the algorithms in the design of methods for detecting overlapping communities in networks. 

Maximum clique detection is often avoided by practitioners from being used as a component in 
a network analysis algorithm on the grounds of its NP-hardness. The results shown here suggest that they need
not be as maximum cliques can in fact be detected rather quickly for most real-world networks that are characterized by sparsity and other structures well suited for branch-and-bound type algorithms.

Our comparison with existing algorithms is understandably not exhaustive.
We have compared against a reasonable sample representatives with emphasis on those
that have publicly available implementations. There are a number of algorithms that do not have publicly available implementation (e.g. \cite{walcom,AAAI101611}) and that would be interesting to compare against in future work. 
%Further, the MCQD implementation uses an adjacency matrix, whereas our algorithm uses an adjacency list to represent the graph. Although it is unlikely for the overall results to be drastically different with a change in the graph representation, it will be interesting to study to what degree the performance will change with
%the change in graph representation.
%The exact algorithm was in general  found to be less successful on relatively dense graphs.
%An interesting line of investigation would be to study ways to overcome this.
%The heuristic's performance is impressive as presented; still it is worthwhile
%to compare with other existing heuristics approaches such as \cite{heu1,heu2}. 
%Another line for future work would be to characterize the class(es) of graphs for which the heuristic is expected to return near-optimal solution.